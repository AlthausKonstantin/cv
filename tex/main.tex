%%%%%%%%%%%%%%%%%
% This is an sample CV template created using altacv.cls
% This fork/modified version has been made by Nicolás Omar González Passerino (nicolas.passerino@gmail.com, 15 Oct 2020)
%
%% It may be distributed and/or modified under the
%% conditions of the LaTeX Project Public License, either version 1.3
%% of this license or (at your option) any later version.
%% The latest version of this license is in
%%    http://www.latex-project.org/lppl.txt
%% and version 1.3 or later is part of all distributions of LaTeX
%% version 2003/12/01 or later.
%%%%%%%%%%%%%%%%

%% If you need to pass whatever options to xcolor
\PassOptionsToPackage{dvipsnames}{xcolor}

%% If you are using \orcid or academicons
%% icons, make sure you have the academicons
%% option here, and compile with XeLaTeX
%% or LuaLaTeX.
% \documentclass[10pt,a4paper,academicons]{altacv}

%% Use the "normalphoto" option if you want a normal photo instead of cropped to a circle
% \documentclass[10pt,a4paper,normalphoto]{altacv}

%% Fork (before v1.6.5a): CV dark mode toggle enabler to use a inverted color palette.
%% Use the "darkmode" option if you want a color palette used to 
% \documentclass[10pt,a4paper,ragged2e,withhyper,darkmode]{altacv}

\documentclass[10pt,a4paper,ragged2e,withhyper]{altacv}

%% AltaCV uses the fontawesome5 and academicons fonts
%% and packages.
%% See http://texdoc.net/pkg/fontawesome5 and http://texdoc.net/pkg/academicons for full list of symbols. You MUST compile with XeLaTeX or LuaLaTeX if you want to use academicons.
\usepackage{fontspec}
\usepackage{varwidth}
\usepackage{fontawesome5}
\usepackage{biblatex}
% Change the page layout if you need to
\geometry{left=1.2cm,right=1.2cm,top=1cm,bottom=1cm,columnsep=0.75cm}

% The paracol package lets you typeset columns of text in parallel
\usepackage{paracol}

% Change the font if you want to, depending on whether
% you're using pdflatex or xelatex/lualatex
\ifxetexorluatex
	% If using xelatex or lualatex:
	\setmainfont{Roboto Slab}
	\setsansfont{Lato}
	\renewcommand{\familydefault}{\sfdefault}
\else
	% If using pdflatex:
	\usepackage[rm]{roboto}
	\usepackage[defaultsans]{lato}
	% \usepackage{sourcesanspro}
	\renewcommand{\familydefault}{\sfdefault}
\fi

% Fork (before v1.6.5a): Change the color codes to test your personal variant on any mode
% \ifdarkmode%
\definecolor{PrimaryColor}{HTML}{836856}
\definecolor{SecondaryColor}{HTML}{545454}
\definecolor{ThirdColor}{HTML}{737373}
\definecolor{BodyColor}{HTML}{ABABAB}
\definecolor{EmphasisColor}{HTML}{836856}
\definecolor{BackgroundColor}{HTML}{F5F3F2}
% \else%
%   \definecolor{PrimaryColor}{HTML}{001F5A}
%   \definecolor{SecondaryColor}{HTML}{0039AC}
%   \definecolor{ThirdColor}{HTML}{F3890B}
%   \definecolor{BodyColor}{HTML}{666666}
%   \definecolor{EmphasisColor}{HTML}{2E2E2E}
%   \definecolor{BackgroundColor}{HTML}{E2E2E2}
% \fi%

\colorlet{name}{PrimaryColor}
\colorlet{tagline}{SecondaryColor}
\colorlet{heading}{PrimaryColor}
\colorlet{headingrule}{ThirdColor}
\colorlet{subheading}{SecondaryColor}
\colorlet{accent}{SecondaryColor}
\colorlet{emphasis}{EmphasisColor}
\colorlet{body}{BodyColor}
\pagecolor{BackgroundColor}

% Change some fonts, if necessary
\renewcommand{\namefont}{\Huge\rmfamily\bfseries}
\renewcommand{\personalinfofont}{\small\bfseries}
\renewcommand{\cvsectionfont}{\LARGE\rmfamily\bfseries}
\renewcommand{\cvsubsectionfont}{\large\bfseries}

% Change the bullets for itemize and rating marker
\renewcommand{\itemmarker}{{\small\textbullet}}
\renewcommand{\ratingmarker}{\faCircle}

\input{bibliography.tex}


\begin{document}
\input{personal_info.tex}
\makecvheader
\color{body}
%% Set the left/right column width ratio to 6:4.
\columnratio{0.25}

% Start a 2-column paracol. Both the left and right columns will automatically
% break across pages if things get too long.
\begin{sloppypar}
	\begin{paracol}{2}
		% ----- TAGS ------
		\input{tags.tex}
		% ----- TAGS ------

		% ----- LANGUAGES -----
		\cvsection{Languages}
		\cvlang{German}{Native}\\
		\divider
		\cvlang{English}{Fluent / C2}
		% ----- LANGUAGES -----

		% ----- REFERENCES -----
		\pagebreak
		\cvsection{References}
		\input{references.tex}

		% ----- REFERENCES -----

		% ----- MOST PROUD -----
		% \cvsection{Most Proud of}
		% \cvachievement{\faTrophy}{Fantastic Achievement}{and some details about it}\\
		% \divider
		% \cvachievement{\faHeartbeat}{Another achievement}{more details about it of course}\\
		% \divider
		% \cvachievement{\faHeartbeat}{Another achievement}{more details about it of course}
		% ----- MOST PROUD -----

		% \cvsection{A Day of My Life}

		% Adapted from @Jake's answer from http://tex.stackexchange.com/a/82729/226
		% \wheelchart{outer radius}{inner radius}{
		% comma-separated list of value/text width/color/detail}
		% \wheelchart{1.5cm}{0.5cm}{%
		%   6/8em/accent!30/{Sleep,\\beautiful sleep},
		%   3/8em/accent!40/Hopeful novelist by night,
		%   8/8em/accent!60/Daytime job,
		%   2/10em/accent/Sports and relaxation,
		%   5/6em/accent!20/Spending time with family
		% }

		% use ONLY \newpage if you want to force a page break for
		% ONLY the current column
		\newpage

		%% Switch to the right column. This will now automatically move to the second
		%% page if the content is too long.
		\switchcolumn

		% ----- ABOUT ME -----
		\cvsection{About Me}
		\begin{quote}
			Lorem ipsum dolor sit amet, consectetur adipiscing elit, sed do eiusmod tempor incididunt ut labore et dolore magna aliqua.
		\end{quote}
		% ----- ABOUT ME -----

		% ----- EXPERIENCE -----
		\cvsection{Experience}
		\input{experience.tex}
		% ----- EXPERIENCE -----

		% ----- PROJECTS -----
		\cvsection{Projects}
		\cvevent{ Junior Actuary }{| ERGO }{ 01/2022 -- 04/2023  }{ Cologne }{ Health Insurance }
\begin{itemize}
\item Automatisation and testing of  actuarial computations in R.
\item Developing GUIs for automated actuarial workflows.
\item Implementing a CI/CD pipeline.
\end{itemize}
\quad\cvtag{ R }\cvtag{ Shiny }\cvtag{ GitHub }\newline
\cvevent{ Research Assistent }{| Technical University  }{ 01/2022 -- 01/2023  }{ Munich }{ Research }
\begin{itemize}
\item Publishing my master’s thesis as a paper in SIAM.
\item Building a CI/CD pipeline for the implemented algorithm.
\end{itemize}
\quad\cvtag{ Python }\cvtag{ GitHub }\newline
\cvevent{ Working Student }{| MEAG }{ 01/2021 -- 01/2022  }{ Munich }{ Asset Management }
\begin{itemize}
\item Automatic parsing and compiling of risk data pertaining to financial assets in R.
\item Automatic visualisation of risk metrics in PowerBi, PowerPoint and HTML reports in R.
\item Creating custom R packages including unit testing and documentation.
\end{itemize}
\quad\cvtag{ R }\cvtag{ Tidyverse  }\cvtag{ Azure DevOps }\newline
\cvevent{ Master’s Thesis }{| Technical University }{ 01/2021 -- 01/2022  }{ Munich }{ Research }
\begin{itemize}
\item Development of a state of the art rare event estimation method.
\item Implementation as a professional Python package.
\item Using containerisation to run numerical experiments on the Google Cloud.
\end{itemize}
\quad\cvtag{ Python }\cvtag{ Docker }\newline
\cvevent{ Joint Research Project }{| Technical University  and Continental Automotive }{ 01/2020 -- 01/2021  }{ Munich }{ Research }
\begin{itemize}
\item Mathematical modelling of an autopilot.
\item Finding the best trajectory through multiple traffic lights.
\item Solving an optimal control problem.
\end{itemize}
\quad\cvtag{ MATLAB }\cvtag{ LaTeX }\newline
\cvevent{ Working Student }{| Cevotec }{ 01/2019 -- 01/2019  }{ Munich }{ Robotics }
\begin{itemize}
\item Implementation and parallelisation of optimisation algorithms in C++.
\item Development of new optimisation methods for fibre patch placement
\item Development of unit and integration tests.
\end{itemize}
\quad\cvtag{ C++ }\cvtag{ Jira }\cvtag{ Bitbucket }\newline
\cvevent{ Research Assistent }{| Technical University  }{ 01/2019 -- 01/2019  }{ Munich }{ Research }
\begin{itemize}
\item Implementation of a time integrator in Julia based on my bachelor’s thesis.
\item Contribution to the scientific computing project  DifferentialEquations.jl
\item Basis for one of the fastest state of the art explicit extrapolation methods.
\end{itemize}
\quad\cvtag{ Julia }\cvtag{ GitHub }\newline
\cvevent{ Internship }{| BMW }{ 01/2018 -- 01/2018  }{ Munich }{ Automotive }
\begin{itemize}
\item Error analysis of coupled systems in MATLAB.
\item Scientific research on the topic of model order reduction.
\item Documentation of research results with LaTeX.
\end{itemize}
\quad\cvtag{ MATLAB }\cvtag{ LaTeX }\newline
		% ----- PROJECTS -----

		% ----- EDUCATION -----
		\begin{breakfreeunit}
			\cvsection{Education}
			\input{education.tex}
		\end{breakfreeunit}
		% ----- EDUCATION -----

		% ----- PUBLICATIONS -----
		\begin{breakfreeunit}
			\cvsection{Publications}
			\nocite{*}
			\printbibliography[heading=pubtype,title={\printinfo{\faFile*[regular]}{Journal Articles}},type=article]
		\end{breakfreeunit}
		% ----- PUBLICATIONS -----
	\end{paracol}
\end{sloppypar}
\end{document}
